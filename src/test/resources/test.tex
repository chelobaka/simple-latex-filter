\pagenumbering{roman}

Hello, world!

Example of \textbf{bold} and \emph{emphasis}.

\chapter{Down the rabbit hole}

Example of closed tag \emph and {virtual group command}.

Let's try a footnote with a hyperlink inside\footnote{\emph{See:} \href{https://wikipedia.org}{Wikipedia}}

We can split \emph{tag pairs between segments.

And it should work} as expected\footnote{Another footnote for tag numbering check.}.

Here is an \unknown{unknown command}.

Next \neverseenit{unknown command}{should}{receive} next numbered tag.

Now let's make empty footnote \footnote{} and emphasis \emph{}

\footnote{External content\par with control\par\\sequence.}

A tag \textbf{\uline{at the beginning}} a of parent tag.

\begin{bar}\begin{foo}Testing environments\end{foo}\end{bar}

% Whole line comment

You should see only this% but not that

\begin{figure}[foo=bar]Option consumer example\end{figure}

\begin{wrapfigure}{123}{boo}Argument consumer example\end{wrapfigure}

Let's test escaping: \% \$ \_ \# \& \{ \} `~` \~{}

URL with escaped characters \url{http://foo.bar?a=1&b=%20%40}

\begin{tabular}{ |p{3cm}||p{3cm}|p{3cm}|p{3cm}|  }
    \hline
    \multicolumn{4}{|c|}{Country List} \\
    \hline
    Country Name or Area Name & ISO ALPHA 2 Code &ISO ALPHA 3 Code & ISO numeric Code\\
    \hline
    Afghanistan & AF & AFG & 004 \\ [1ex]
    \hline
\end{tabular}

Inline math types: \(E=mc^2\), $E=mc^2$, \begin{math}E=mc^2\end{math}.

%Display math flavors
\[E=mc^2\] \begin{displaymath}E=mc^2\end{displaymath} \begin{equation}E=mc^2\end{equation}

% verb and url behind a comment \verb!something! and \url{http://site.com/?a=%0203}

\begin{verbatim}This is verbatim \emph{text} % Not a comment \verb+HERE+\end{verbatim}

A `verb` command test: \verb+This~is~unescaped+